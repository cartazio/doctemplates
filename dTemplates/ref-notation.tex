% thm style defaults

\theoremstyle{plain}
\newtheorem{theorem}{Theorem}[section]
\newtheorem{corollary}[theorem]{Corollary}
\newtheorem{lemma}[theorem]{Lemma}
\newtheorem{claim}[theorem]{Claim}

\theoremstyle{definition}
\newtheorem{conjecture}{Conjecture}[section]
\newtheorem{definition}{Definition}[section]
\newtheorem{example}{Example}[section]

\theoremstyle{remark}
\newtheorem{observation}{Observation}
\newtheorem{remark}{Remark}

%labelling aids
\newcommand{\PropositionName}[1]{\label{prop:#1}}
\newcommand{\LemmaName}[1]{\label{lma:#1}}
\newcommand{\TheoremName}[1]{\label{thm:#1}}
\newcommand{\FigureName}[1]{\label{fig:#1}}

\newcommand{\Proposition}[1]{Proposition~\ref{prop:#1}}
\newcommand{\Lemma}[1]{Lemma~\ref{lma:#1}}
\newcommand{\Theorem}[1]{Theorem~\ref{thm:#1}}
\newcommand{\Figure}[1]{Figure~\ref{fig:#1}}